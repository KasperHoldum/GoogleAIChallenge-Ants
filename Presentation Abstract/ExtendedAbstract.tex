\documentclass[a4paper]{llncs}

\usepackage{graphicx}
\usepackage{algorithm}
\usepackage{algorithmic}
\usepackage{hyperref}

\begin{document}

\author{Kasper Hjorth Holdum \and Christian Kays{\o}-R{\o}rdam \and Christopher {\O}stergaard de Haas}


\title{Google AI Challenge 2011: Ants}

\institute{DTU Informatics}
\date{December 2011}
\maketitle
\thispagestyle{plain} \pagestyle{plain}

\smallskip

\begin{abstract}
This year's Google AI Challenge is a multiplayer game on a 2-dimensional map. All players are given a short time frame each turn to do calculation and issue orders to all owned units. This paper identifies the key problems for a well performing AI system. Solutions to each problem are proposed and discussed.

\end{abstract}

\medskip

\section{Introduction}

Google AI Challenge is an open annual contest, hosted on http://aichallenge.org/. When the annual challenge is announced contestants have a few months to create, modify and perfect a bot. A bot is an AI system to control your ants. During the contest, contestants can upload their bot to the contest servers where they will play against other bots and be given a running rank according to performance. The contest supports any language that can write to the standard I/O. These conditions make a perfect environment for a large varity of people, strategies and skill levels.

We will take a look at which solutions we have found for each of the sub problems. 

\section{Problem Description}
This years Google AI Challenge is called Ants, which is a turn-based AI game. Each turn an ant may move up, down, left or right on the grid-based map. The map is shaped like torus, meaning there are no edges on the map. Maps may include obstacles, which cannot be directly passed. Each player has one or more hills, on which new ants spawn. Additional ants are acquired by collecting food, which is randomly spawned on the map. Ants can attack enemy ants or hills, where only attacking a hill rewards points. The visibility is limited for each player. Each ant provides vision for the player in a constant radius around the ant. There exists several conditions in which the game will end. A few noticable of these are when a certain turn limit is hit, or when all enemy ants are eliminated. The game is won by the player who gathered the most points.

Each turn every bot is given 500ms to do all calculations and issue orders for all its ants. The time limit imposed on all bots greatly alters the problem from finding optimal solutions for each sub problem, which will likely take much longer than the given time limit. Instead we will have to find approximate solutions that are as optimal as possible while staying within the time limit. The most significant problems for the bot are reproduction, exploration, attacking enemy ants/hills and defending hills.
\section{Tools}
During development of our bot we make use of several tools to assist us in debugging and estimating the quality of the current algorithms. The host of the challenge provides two tools: An application that will run a single match with test bots and/or bots given as input and a visualizer that will display the match turn by turn after it has ended. While both of these are a tremendous help, they do restrict the development a bit. It is not possible to start a bot in our IDE and expect to debug it while a match is playing. Neither is it possible to do any custom visualization. Furthermore, since it is only possible to play against simple test bots and your own bots it is hard to estimate the quality of the gameplay delivered by the bot.

It is possible to upload a bot to the contest page after which it will start to play games against the other uploaded bots. However, due to the large amount of contestants and low amount of server capacity this is a very slow process. At the time of this writing each bot plays about 3-4 games every day. Several people decided to create their own servers and ranking systems where you can connect via TCP. The advantage of these servers are that each bot owner uses the processing power of the computer from which the bot is connected. Since the server no longer has to use it's CPU cores for the bots, it can now host a much larger amount of games. So by utilizing the TCP servers it is possible to play games against real players much faster.

\subsection{HPA Visualizer}
We implemented the hierarchyal pathfinding algorithm [BMS2004] to improve the running time of path finding. HPA will be covered under in section \ref{sec:datastructures}. Due to the involved nature of the implementation of HPA, we decided to make a visualizer for debugging. It shows us the map split into the zones and the transition points for each of the zones, and we may ask for a path from one point to another and have it shown in the visualizer as well. Being able to see the actual path generated is a big help, as the path is likely to not be optimal and being able to clearly see what causes this sub optimality is useful when we will design the path smoothing for HPA. Path smoothing will be covered in section \ref{sec:datastructures}

A screenshot of the visualizer can be seen in the appendix, fig \ref{fig:awesome_image}.
\section{Solution}
We have identified several key problems in the challenge:
\begin{itemize}
\item Spreading out ants,
\item finding and eating food,
\item attacking enemy hills,
\item and defending own hills.
\end{itemize}
We will discuss each of these problems and our proposed solution in the following sections. Since each turn is limited in time to 500 ms (subject to change) all algorithms and solutions must be designed with this limit in mind. For more complex solutions we have been forced to implement advanced data structures to minimize the running time. The data structures are briefly mentioned section~\ref{sec:datastructures}.

\subsection{Spreading Out Ants}
The final objective of the game is to eliminate as many enemy hills as possible. To achieve this goal we need to know the locations of the hills and have a large number of ants to attack them. Both require that we explore the map. Since food randomly spawns on the entire map (taking symmetry into account) we would optimally want to have sight of every single field on the map, and minimize the distance from our closests ant to each field. This is not possible, at least in the early game, due to lack of ants.

We have tried two approaches. One that focused on which part of the map that is not currently visible to us, and the other which ignores visibility and simply uses a simple model of representing ants as magnetic entities that repel each other.

\subsubsection{Visibility Spread Out}
Our first approach is to send every ant to the spot closest to that ant which is not visible. This approach requires information about which fields are visible as well as a method to find the nearest invisible field and a method to find invisible fields in a certain radius of a field. These two operations are provided by the Kd-tree and covered in section~\ref{sec:datastructures}.

The basic algorithm iterates over all the available ants (ants that has not been occupied elsewhere), finds the closest invisible spot, tries to find a path to this spot and then issues an order for the ant to move there.


\begin{algorithm}
\caption{Visibility Spread Out}
\label{alg:visibilityspreadout}
\begin{algorithmic}
\FOR{$i = 1 \to count(availableAnts)$} 
\STATE $currentAnt \gets availableAnts[i]$
\STATE $closestInvisibleSpot \gets findClosestInvisSpotTo(currentAnt)$
\STATE $pathToGoal \gets findPath(currentAnt, closestInvisibleSpot)$


\IF{pathToGoal != null}
\STATE $moveAnt(currentAnt, pathToGoal.nextStep)$
\ENDIF
\ENDFOR
\end{algorithmic}
\end{algorithm}

The algorithm has been modified afterwards to improve several aspects. First of all, we have added caching so, that if an ant is still heading for the same location, then the whole path should not be recalculated. This issue is trickier than first expected though, since parts of the map might be discovered and invalidate the path. Likewise, ants might get in the way of the calculated path. However, once we finish the HPA data structure, this should (hopefully) no longer be an issue.

The other issue, is the way the closest invisible spot is selected. Instead of just selecting the closest invisble spot, we have had success with instead selecting the closest invisible spot which has not been seen for some amount of turns.

\subsubsection{Magnetic Spread Out}
If an ant has nothing to do, we want it to go explore. But since an ant may only move a single tile per turn, generating and storing a path to some destination becomes slow and ineffective, as many events may occur prior to that ant reaching its destination, making it a bad destination. Generating paths can also be very time consuming, so we wanted a way to spread out our ants without using too much time on it.

The idea is quite simple; we make all our ants repel each other if they are in close proximity and unlike actual magnets they never attract each other. This repulsion should somehow scale with the distance between two ants, to allow some ants to be squeezed closer together. This will cause all the ants to eventually end up in an evenly spaced grid, covering most of the map, giving us as much information about food and enemy locations as possible. When we get new ants from eating food, the grid will automatically adjust and expand to match the new number of ants.

The algorithm for this looks as follows:

\begin{algorithm}
\caption{Magnetic Spread Out}
\label{alg:magneticspreadout}
\begin{algorithmic}
\STATE $antForce \gets 10$
\FOR{$i = 1 \to count(availableAnts)$} 
	\STATE $currentDirection \gets (0,0)$
	
	\FOR{$j = 1 \to count(antsInRange)$}
		\STATE $scaledForce \gets antForce / distanceTo(ant[i],ant[j])$
		\STATE $directionToJ \gets getVector(ant[j],ant[i])$
		\STATE $currentDirection \gets currentDirection + directionToJ * scaledForce$
	\ENDFOR

	\STATE $currentDirection \gets normalize(currentDirection)$
	\STATE $moveAntInDirection(ant[i], currentDirection)$
\ENDFOR
\end{algorithmic}
\end{algorithm}

For assigning a direction for all our ants the running time of this approach can be as high as $O(N^2)$ (where N is the number of ants), which happens if all the ants were clumped up next to each other.

Once the ants were able to behave in a fashion similar to magnets, we wanted to try and see if it was possible to make them perform other strategies by further extending the magnet concept. Since the repulsion is based on a positive number, we could easily add an auxilary "ant", or attraction point, who attracted other ants. Such attraction points could then be placed on food locations or in narrow choke points to safe guard an area. This does cause a few problems though, as many ants could potentially go for a single food and then end up repelling each other such no ant ever reached the food. For guarding choke points we would face much the same problem as we would simply have an area with high density of ants but this does not ensure that any of the ants are safe from attackers.
\subsection{Finding Food}
Food spawns randomly on the map at the beginning of every turn. Being able to effeciently collect food and build up an army of ants is one the key aspects of a good bot. Half of the work is done by initially exploring the map and thereby finding the food.

Once the food has been discovered the bot needs to collect it by sending one or more ants to eat it. In our initial and current algorithm for deciding which ants should collect which food we use the stable marriage algorithm [GS1962] though slightly modified. Each pair of ant/food has the same preference for each other, however, we only form the pair where the distance between them is lesser than twice the view radius of the ant. This is a performance optimization. After we have found the ant/food pairs we find the path between every ant and its food and order the ant to start moving along it.

Enemy ants close to the food potentially pose a problem. If they are closer to the food than our closest ant, then we might as well give up, since it will always reach the food before us given that taking the food is the optimal solution and he plays optimally. If we are equally far from the food, then we have the possibility of sending several ants. Both of these considerations have not yet been implemented, but still remain as viable options for further improvement.

Furthermore, when two or more pieces of food are close to each other, it would make sense to only send a single ant to collect all of them.  
\subsection{Attacking Enemy Hills and Ants}
Attacking is a hard and very important problem for the bot. Given the grid-based movement, we must rely on very precise collaboration of two or more ants in order to carry out a successful attack. We must have at least two ants attacking a single enemy ant simultaneously in order to avoid sacrificing an ant. Sacrificing ants is certainly an option, but it is associated with defense, not offence.

Correctly placing ants around an enemy ant, in order to attack it, requires precise timing. Furthermore it involves some prediction of the enemy ant's future movement. As ants can attack each other within a given radius (read circle), we cannot guarantee sacrificing one for one when attacking with just two ants. We can prevent losses in 3 of 4 cases of enemy ant movement. Using three ants in an attack, we can ensure no losses of our own, yet there is still the possibility of the enemy ant fleeing. All of this requires that only one enemy ant is within attack radius. This is mostly fine, as an attack on a single enemy ant is likely due to collecting food, otherwise we will be attacking an enemy hill which obviously cannot move.

Attacks require a number of ants doing basically the same thing for a number of turns. During this time, these ants will not be collecting food, thus reproducing further ants. This makes it hard to determine when an attack is appropriate, and just as important, which ants to use. As the game goes on and the density of ants increase this quickly becomes a complex and time-consuming problem to solve. For the current version of the bot, we rely on only looking at a small subset of closely located ants in order to carry out an attack. We are not yet sure how to prioritize attacks vs. finding food or exploring the map. 
\subsection{Hill Defence}
When an ant moves on top of an enemy hill, the hill is destroyed and can no longer produce ants. Furthermore the owner loses two points.

We work with two zones of danger: Very close and thus dangerous and medium close, possibly dangerous. If an enemy ant is in very close proximity to one our hills, we assign the closest ant to move next to this ant which leads to a one to one trade off. If enemy ants are in medium range we assign an ant to move close to our hill for each enemy ant in this zone.

\begin{algorithm}
\caption{Simple Hill Defence}
\label{alg:simplehilldefence}
\begin{algorithmic}
\FOR{$i = 1 \to count(myHills)$} 
\STATE $currentHill \gets myHills[i]$
\STATE $dangerousAnts \gets enemyAnts.FindInRange(currentHill, dangerThresholdRange)$
	\FOR{$j = 1 \to count(dangerousAnts)$} 
	\STATE $nearestAvailableAnt \gets myAvailableAnts.FindNearest(currentHill)$
	\STATE $goal$

	
	\IF{$distanceTo(dangerousAnts[j], currentHill) > criticalThreshold$}
		\STATE $goal \gets dangerousAnts$
	\ELSE
		\STATE $goal \gets myHill$
	\ENDIF

	\STATE $pathToGoal \gets findPath(nearestAvailableAnt, goal)$
	\STATE $moveAntAlongPath(nearestAvailableAnt, pathToGoal)$
	\ENDFOR
\ENDFOR
\end{algorithmic}
\end{algorithm}
\subsection{Data Structures}
\label{sec:datastructures}
During the development of the bot we have actively used common data structures such as hash arrays and dynamic arrays. Besides these we have implemented three data structures to minimize the running time of our algorithms.

The first, SortList, is a dynamic array which allows us to insert elements and then it keeps them sorted in accordance to a certain logic which is provided when the data structure is created. SortList sorts itself by utilizing a binary search every time a member is inserted. We use this data structure when performing A-star searches to keep unevaluated nodes sorted ascendingly by distance to goal. This data structure did not exist in our language, and is otherwhere known as a priority queue.

Very often our bots has the need for "find nearest neighbour"- and "find nodes in range x"-searches. These are not trivial in 2-dimensional wrapping space. The Kd-Tree data structure provides these operations, and we have implemented it and used it with success so far. However, the Kd-tree has a fundamental flaw: It is not designed to work with spaces that wrap at the borders. We have not yet found a solution to this problem, but we have considered switching to a quad-tree.

During the competition the time limit for each turn was reduced from 2 seconds to 500 ms which caused our bot to time out. The A-star path finding algorithm used almost 85\% of the total running time each turn when we had 80-140 ants. The main issue was that we calculated the full path between an ant and its goal every single turn. We tried caching these calculated paths between turns, but since an exploring ant could be disrupted by a spawned piece of food this was not enough when the number of ants grew. To solve this issue we decided to search for data structures/algorithms that would allow us to only compute a partial path and possibly allowing us to cache paths between popular locations. The partial paths found the by the data structure, is the path from the current location, to the nearest transit point - or to goal if it is within the same cluster.  Hierachical Path-Finding using A-star solves both of these issues. This solution divides the space into a grid, and then only computes partial paths between these grids. We are still in the process of implementing this data structure.
\section{Conclusion}
Designing an AI for the challenge has proven to be difficult. Especially the time constraint had great influence on the algorithms. The challenge ends on December 18th 2011 and we still have a lot of ideas for improvements. We are currently ranked 166th of approximately 6000 players.



\begin{thebibliography}{1}

\bibitem{BMS2004}
A. Botea, M. M\"{u}ller, and J. Schaeffer.  
\newblock Near Optimal Hierarchical Path-Finding.
\newblock Journal of Game Development, vol. 1, no. 1, 7-28, 2004.
\bibitem{KT2006}
J. Kleinberg and Eva Tardos.
\newblock Algorithm Design.
\newblock Pearson International Edition 2006.
\bibitem{GS1962}
D. Gale and L. S. Shapley.
\newblock College Admissions and the Stability of Marriage.
\newblock American Mathematical Monthly 69, 9-14, 1962.

\bibitem{Bent1975}
J. L. Bentley.
\newblock Multidimensional binary search trees used for associative searching.
\newblock Communications of the ACM, 18(9):509-517, 1975.

\bibitem{Sharat}
Chandran, Sharat.
\newblock Introduction to kd-trees.
\newblock University of Maryland Department of Computer Science, http://www.cs.umd.edu/class/spring2002/cmsc420-0401/pbasic.pdf, 2002.
\end{thebibliography}
\section*{Appendix}
\begin{figure}[htb]
\centering
\includegraphics[keepaspectratio, width=350px]{Tools/hpavisualizer.png}
\caption{A screenshot of the HPA visualizer. Green background is walkable terrain, dark blue is water and unwalkable. The clusters are visualized with black rectangles. The red rectangles are the transit points between clusters. The blue line between transit points are the internal cluster connections. }
\label{fig:awesome_image}
\end{figure}

\end{document}
