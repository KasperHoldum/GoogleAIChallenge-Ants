\subsection{Data Structures}
\label{sec:datastructures}
During the development of the bot we have actively used common data structures such as hash arrays and dynamic arrays. Besides these we have implemented three data structures to minimize the running time of our algorithms.

The first, SortList, is a dynamic array which allows us to insert elements and then it keeps them sorted in accordance to a certain logic which is provided when the data structure is created. SortList sorts itself by utilizing a binary search every time a member is inserted. We use this data structure when performing A-star searches to keep unevaluated nodes sorted ascendingly by distance to goal. This data structure did not exist in our language, and is otherwhere known as a priority queue.

Very often our bots has the need for "find nearest neighbour"- and "find nodes in range x"-searches. These are not trivial in 2-dimensional wrapping space. The Kd-Tree data structure provides these operations, and we have implemented it and used it with success so far. However, the Kd-tree has a fundamental flaw: It is not designed to work with spaces that wrap at the borders. We have not yet found a solution to this problem, but we have considered switching to a quad-tree.

During the competition the time limit for each turn was reduced from 2 seconds to 500 ms which caused our bot to time out. The A-star path finding algorithm used almost 85\% of the total running time each turn when we had 80-140 ants. The main issue was that we calculated the full path between an ant and its goal every single turn. We tried caching these calculated paths between turns, but since an exploring ant could be disrupted by a spawned piece of food this was not enough when the number of ants grew. To solve this issue we decided to search for data structures/algorithms that would allow us to only compute a partial path and possibly allowing us to cache paths between popular locations. The partial paths found the by the data structure, is the path from the current location, to the nearest transit point - or to goal if it is within the same cluster.  Hierachical Path-Finding using A-star solves both of these issues. This solution divides the space into a grid, and then only computes partial paths between these grids. We are still in the process of implementing this data structure.