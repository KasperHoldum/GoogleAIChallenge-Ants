\section{Tools}
During development of our bot we make use of several tools to assist us in debugging and estimating the quality of the current algorithms. The host of the challenge provides two tools: An application that will run a single match with test bots and/or bots given as input and a visualizer that will display the match turn by turn after it has ended. While both of these are a tremendous help, they do restrict the development a bit. It is not possible to start a bot in our IDE and expect to debug it while a match is playing. Neither is it possible to do any custom visualization. Furthermore, since it is only possible to play against simple test bots and your own bots it is hard to estimate the quality of the gameplay delivered by the bot.

It is possible to upload a bot to the contest page after which it will start to play games against the other uploaded bots. However, due to the large amount of contestants and low amount of server capacity this is a very slow process. At the time of this writing each bot plays about 3-4 games every day. Several people decided to create their own servers and ranking systems where you can connect via TCP. The advantage of these servers are that each bot owner uses the processing power of the computer from which the bot is connected. Since the server no longer has to use it's CPU cores for the bots, it can now host a much larger amount of games. So by utilizing the TCP servers it is possible to play games against real players much faster.

\subsection{HPA Visualizer}
We implemented the hierarchyal pathfinding algorithm [BMS2004] to improve the running time of path finding. HPA will be covered under in section \ref{sec:datastructures}. Due to the involved nature of the implementation of HPA, we decided to make a visualizer for debugging. It shows us the map split into the zones and the transition points for each of the zones, and we may ask for a path from one point to another and have it shown in the visualizer as well. Being able to see the actual path generated is a big help, as the path is likely to not be optimal and being able to clearly see what causes this sub optimality is useful when we will design the path smoothing for HPA. Path smoothing will be covered in section \ref{sec:datastructures}

A screenshot of the visualizer can be seen in the appendix, fig \ref{fig:awesome_image}.